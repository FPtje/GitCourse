\documentclass[10pt,a4paper]{beamer}
\usepackage[utf8]{inputenc}
\usepackage{amsmath}
\usepackage{amsfonts}
\usepackage{amssymb}
\usepackage[labelformat=empty]{subcaption}
\usepackage{graphicx}
\usepackage[dutch]{babel}
\usepackage{lmodern}
\usepackage{epstopdf}
\usepackage{ulem}
\usepackage{graphicx}
\usepackage[labelformat=empty]{caption}

\author{Falco Peijnenburg}
\usetheme{Warsaw}
%\setbeamertemplate{footline}{\insertframenumber/\inserttotalframenumber}


\title[Git crash course\hspace{40mm} \insertframenumber/\inserttotalframenumber]{Git crash course}


\begin{document}

% Title frame
\frame{\titlepage}

\setcounter{tocdepth}{1}
% Table of contents
\begin{frame}
\frametitle{Table of contents}
\tableofcontents[]
\end{frame}



\section{Advantages git}

\subsection{By itself}
\begin{frame}{By itself}
\begin{itemize}
\item Distributed
\item Fast (both in speed of commands and workflow)
\item You can work offline
\item Forces you to have proper, structured workflow (and \textit{yes} this is good for everyone)
\item Has many features that are actually useful
\item It's useful even for people who work alone
\item Amazing branching implementation

\end{itemize}
\end{frame}
% multiple remotes, no problem

\subsection{Compared to Dropbox/Google Drive}
\begin{frame}{Compared to Dropbox/Google Drive}
\begin{itemize}
\item No more copying folders for an experimental feature/recode (seriously, this is ridiculous)
\item Work simultaneously without the \textit{utter} chaos of conflict files
\item You can very easily revert if you make a mistake
\item Better tracking of changes
\item The ability to find out when and by whom a bug was introduced
\end{itemize}
\end{frame}

\subsection{Compared to SVN}
\begin{frame}{Compared to SVN}
\begin{itemize}
\item \textit{Loads} faster
\item Branching isn't terrible
\item Cli isn't terrible
\item SVN is \textit{very} limited in features once you know what git can do
\item You can have multiple remotes (will be explained later)
\item Pull requests
\item Git has no single point of failure.
\item Integrity checking (minor)

\end{itemize}
\end{frame}


\section{Getting started with git}


\subsection{How to install}
\begin{frame}{How to install}
\begin{itemize}
\item Linux:
\begin{itemize}
\item apt-get install git
\item pacman -S git
\item yum install git
\item ...
\end{itemize}
\item Windows/Mac: http://git-scm.com/downloads

When installing, use advanced context menu integration. Use default settings in other screens
\end{itemize}
\end{frame}

\subsection{Opening git bash}
\begin{frame}{Opening git bash}
Linux/Mac
\begin{itemize}
\item Open the terminal
\item cd to the right folder
\end{itemize}

Windows:
\begin{itemize}
\item Right click a folder
\item Click ``Git Bash''. Don't use the normal command prompt.
\end{itemize}

Tip: Leave the terminal open in the background while you work.


\end{frame}

\subsection{Setting up}
\begin{frame}[fragile]{Setting up}
\begin{itemize}
\item Git needs to know your name and email address.
\item It uses that information to assign an author to a commit.
\end{itemize}

\begin{verbatim}
git config --global user.name "John Doe"
git config --global user.email "johndoe@example.com"
\end{verbatim}

\begin{itemize}
\item You will need a public/private keypair to communicate with a git server through ssh. Follow this guide:
\item \url{https://help.github.com/articles/generating-ssh-keys}
\end{itemize}
\end{frame}

\subsection{Making a repository}
\begin{frame}[fragile]{Making a repository}
Two options:

\begin{itemize}
\item Start your own empty local repository.
\end{itemize}

	\begin{verbatim}
	git init
	\end{verbatim}

\begin{itemize}
\item Clone the repository from a server.
\end{itemize}

	\begin{verbatim}
	git clone URL_HERE
	git clone git@github.com:FPtje/DarkRP.git
	\end{verbatim}
\end{frame}


\section{Basic git structure}
\subsection{Basic git structure}
\begin{frame}{Basic git structure}
A git repository has three states
\begin{itemize}
\item Working directory
\item Staging area
\item Git directory
\end{itemize}
\end{frame}

\begin{frame}[plain]
\includegraphics[width=\linewidth]{threeStates.png}
\end{frame}

\begin{frame}[plain]
\includegraphics[width=\linewidth]{fileLifeCycle.png}
\end{frame}

\subsection{Initial commit}
\begin{frame}{Initial commit}
Applies only to an empty repository (created by git init)
\includegraphics[width=\linewidth]{gitStatus1.png}
\end{frame}

\subsection{Adding/removing files}
\begin{frame}
\includegraphics[width=\linewidth]{gitAdd.png}
\end{frame}

\begin{frame}
\includegraphics[width=\linewidth]{gitrmcached.png}
\end{frame}

\begin{frame}
\includegraphics[width=\linewidth]{gitaddall.png}
\end{frame}

\begin{frame}
There's also:
\begin{itemize}
\item git mv file location -- Move or rename a file. Same syntax as mv command.
\item .gitignore -- a file that contains regex patterns of files that git should leave alone. Use this for executables, .obj files etc.
\end{itemize}
Tip for .gitignore files:
\url{https://github.com/github/gitignore}
\end{frame}

\subsection{Committing changes}
\begin{frame}[fragile]{Committing changes}
\begin{itemize}
\item Commit everything that's staged using git commit
\item Every commit message \textit{must} have a commit message! Tell git what you've done!
\end{itemize}
\begin{verbatim}
-- Opens vi or nano by default for the commit message:
git commit
-- Inline commit message:
git commit -m "Message here"
-- Skip staging of modified files
git commit -a -m "Message here"
git commit -am "Message here"
\end{verbatim}
\begin{itemize}
\item You cannot skip the staging of new files. Add them manually.
\end{itemize}
\end{frame}


\section{Further commits}

\subsection{git diff}
\begin{frame}
\includegraphics[width=\linewidth]{gitdiff.png}
\end{frame}

\subsection{Git checkout file}
\begin{frame}

\end{frame}

\subsection{More on git commit}
\begin{frame}

\end{frame}

\subsection{git log}
\begin{frame}

\end{frame}

\subsection{git show}
\begin{frame}

\end{frame}


\section{The tree of commits}

\subsection{Master branch}
\begin{frame}

\end{frame}

\subsection{Creating your own branch}
\begin{frame}

\end{frame}

\subsection{Switching branches}
\begin{frame}

\end{frame}

\subsection{Merging branches}
\begin{frame}

\end{frame}

\subsection{Removing branches}
\begin{frame}

\end{frame}


\section{Resolving conflicts}

\subsection{What is a conflict}
\begin{frame}

\end{frame}

\subsection{Fixing the conflicts in the file}
% It's not a mess
\begin{frame}

\end{frame}

\subsection{Registering merged file}
\begin{frame}

\end{frame}

\subsection{Committing the merge}
\begin{frame}

\end{frame}


\section{Remote repositories}

\subsection{Adding a remote repository}
\begin{frame}

\end{frame}

\subsection{Showing remote repositories}
\begin{frame}

\end{frame}

\subsection{Removing a remote repository}
\begin{frame}

\end{frame}

\subsection{Pushing and pulling}
\begin{frame}

\end{frame}


\section{Best Practices and tips}
\begin{frame}

\end{frame}


\section{Awesome things you can do with git}

\subsection{Git revert}
\begin{frame}

\end{frame}

\subsection{Submodule/Subtree merge}
\begin{frame}

\end{frame}

\subsection{Git bisect}
\begin{frame}

\end{frame}

\subsection{Git tag}
\begin{frame}

\end{frame}

\subsection{Git stash}
\begin{frame}

\end{frame}

\subsection{Git clean}
\begin{frame}

\end{frame}

\subsection{Git blame}
\begin{frame}

\end{frame}

\subsection{Pull requests}
\begin{frame}

\end{frame}

\subsection{Cherry-pick}
\begin{frame}

\end{frame}

\subsection{Set up your own git server}
\begin{frame}

\end{frame}

\section{Further reading}
\begin{frame}

\end{frame}

\end{document}